\section{Introduction}

\subsection{RoboJackets}

%stuff about club and teams
RoboJackets is a competitive robotics organization at the Georgia Institute of Technology. Founded in 1999 as a BattleBots team, the organization has since grown to include RoboCup Small Size League, Intelligent Ground Vehicle Competition, and a large outreach team. While chartered in the school of mechanical engineering, members come from all departments (predominately computer science, mechanical engineering, aerospace engineering, and electrical engineering) to participate in these extracurricular activities. First competing in the IGVC in 2004, the RoboJackets have competed in every IGVC since 2006, finishing in the top 10 on the autonomous course since 2007.

\subsection{Team Members}

This year's team was unusually young, consisting largely of freshman with some senior and graduate leadership. The members participating on this project are listed in Table \ref{TAB:RJTeam} As such this year was treated as an opportunity to rejuvenate the team with a completely new design and software architecture.

\begin{table}[H]
\begin{center}
\caption{2012 RoboJackets / IGVC Team}
\begin{tabular}{| l | p{2.4in} | p{2in} |}
\hline
Name & Degree / Class & Role\\ \hline
Matt Barulic &		BS Computer Science / Freshman& Software\\ \hline
Kyle Bates & 		BS Electrical Engineering / Freshman& Software\\ \hline
Al Chaussee &		BS Computer Science / Freshman& Software\\ \hline
William Evans &		BS Mechanical Engineering / Senior& Mechanical Design and Build\\ \hline
Joseph Hickey &		BS Mechanical Engineering / Senior& Project Manager and	Mechanical Lead\\ \hline
Emanuel Jones &		MS Mechanical Engineering& Mechanical Design and Build\\ \hline
Dea Gyu Kim &	BS Mechanical Engineering / Freshman & Mechanical Design and Build\\ \hline
Nikolaus Mitchell &		BS Mechanical Engineering / Junior& Mechanical Build\\ \hline
Alex Trimm &	BS Computer Science / Senior&	Software Lead, Electrical Lead\\ \hline
Jonathan Williams &	BS Mechanical Engineering / Freshman & Mechanical Design and Build\\ \hline


\end{tabular}
\label{TAB:RJTeam}
\end{center}
\end{table}

The team was organized into mechanical and software subteams, with electrical being a collaborative effort between the subteams. The beginning of fall semester was spent overviewing previous entries into IGVC, forming design requirements, and beginning conceptual design. Fabrication began in the spring semester, completing during the beginning of summer. In total approximately 2500 man hours were spent over the course of the 2012-2013 academic year.
